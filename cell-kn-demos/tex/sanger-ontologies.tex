\documentclass[aspectratio=169,xcolor=dvipsnames]{beamer}

\mode<presentation>
{
  \usetheme{default}

  \usecolortheme{default}
  \usefonttheme{default}
  \useinnertheme{circles}

  \setbeamertemplate{navigation symbols}{}
  \setbeamertemplate{caption}[numbered]

  \setbeamertemplate{title page}[default][center]
  \setbeamercolor{title}{fg=black}
  \setbeamercolor{subtitle}{fg=black}
  \setbeamercolor{author}{fg=black}
  \setbeamercolor{date}{fg=black}

  \setbeamertemplate{frametitle}[default][leftskip=1.0cm]
  \setbeamercolor{frametitle}{fg=black}
  
  \setbeamertemplate{footline}[text line]{\color{white}\raisebox{9.75pt}{\hspace{-25pt}\insertframenumber/\inserttotalframenumber}}
}

\usepackage{amsmath}
\usepackage{amssymb}
\usepackage[english]{babel}
\usepackage{hyperref}
\usepackage{listings}
\usepackage{mathrsfs}
\usepackage[mathscr]{euscript}
\usepackage{pifont}
\usepackage[utf8x]{inputenc}
\usepackage[normalem]{ulem}

\newcommand{\cmark}{\ding{51}}
\newcommand{\xmark}{\ding{55}}

\hypersetup{colorlinks=true,citecolor=black,linkcolor=black,urlcolor=blue}

\title[]{Ontologies Loaded by Sanger}
\subtitle[]{IDs, Descriptions, Counts by Triple Type, and Size}
\author{Raymond LeClair}
\date{Working Copy of February 3, 2025}

\begin{document}

{
  \usebackgroundtemplate{\includegraphics[width=\paperwidth]{NLM-2024-Title-Background.png}}
  \setbeamertemplate{footline}{}
  \begin{frame}
    \vspace{6\baselineskip}
    \titlepage
  \end{frame}
}

\setcounter{framenumber}{0}

\usebackgroundtemplate{\includegraphics[width=\paperwidth,height=\paperheight]{NLM-2024-Slide-Background.png}}

% \begin{frame}{Outline}
%   \tableofcontents[sectionstyle=show,subsectionstyle=hide]
% \end{frame}

% \begin{frame}{Outline}
%   \tableofcontents[sectionstyle=show/shaded,subsectionstyle=show/shaded/hide]
% \end{frame}

\begin{frame}
  \frametitle{Ontology IDs and Descriptions (1 of 2)}
  \framesubtitle{}
  \begin{itemize}\footnotesize
  \item[\href{"http://purl.obolibrary.org/obo/cl.owl"}{cl}]: Cell
    Ontology ... a structured controlled vocabulary for cell types in
    animals
  \item[\href{"http://purl.obolibrary.org/obo/pcl.owl"}{pcl}]:
    Provisional Cell Ontology [describes] cell types that are
    provisionally defined by experimental techniques such as single
    cell or single nucleus transcriptomics rather than a
    straightforward \& coherent set of properties
  \item[\href{"https://raw.githubusercontent.com/Cellular-Semantics/CellMark/refs/heads/main/clm-kg.owl"}{clm-kg}]:
    A data-linked knowledge graph for the Cell Ontology
  \item[\href{"https://purl.obolibrary.org/obo/go/extensions/go-plus.owl"}{go}]:
    Gene Ontology ... for describing the function of genes and gene
    products
  \item[\href{"http://purl.obolibrary.org/obo/uberon/uberon-base.owl"}{uberon}]:
    Uberon multi-species anatomy ontology ... covering animals and
    bridging multiple species-specific ontologies
  \item[\href{"http://purl.obolibrary.org/obo/ncbitaxon/subsets/taxslim.owl"}{ncbitaxon}]:
    An ontology representation of the NCBI organismal taxonomy, the
    slim is intended to cover:
    \begin{itemize}\scriptsize
    \item Anything used in a taxon constraint in an ontology
    \item All UniProt Reference Proteomes
    \item Any taxon that has a non-IEA annotation in GO
    \end{itemize}
  \end{itemize}
\end{frame}

\begin{frame}
  \frametitle{Ontology IDs and Descriptions (2 of 2)}
  \framesubtitle{}
  \begin{itemize}\footnotesize
  \item[\href{"http://purl.obolibrary.org/obo/mondo/mondo-simple.owl"}{mondo}]:
    Mondo Disease Ontology a global community effort to harmonize
    multiple disease resources to yield a coherent merged ontology
  \item[\href{"http://purl.obolibrary.org/obo/pato.owl"}{pato}]:
    Phenotype And Trait Ontology ... of phenotypic qualities
    (properties, attributes or characteristics)
  \item[\href{"http://purl.obolibrary.org/obo/hancestro/hancestro.owl"}{hancestro}]:
    Human Ancestry Ontology ... provides a systematic description of
    the ancestry concepts used in the NHGRI-EBI Catalog of published
    genome-wide association studies
  \item[\href{"http://purl.obolibrary.org/obo/mmusdv.owl"}{mmusdv}]:
    Mouse Developmental Stages [describes] life cycle stages for Mus
    Musculus
  \item[\href{"http://purl.obolibrary.org/obo/hsapdv.owl"}{hsapdv}]:
    Human Developmental Stages [describes] life cycle stages for Human
  \item[\href{"https://raw.githubusercontent.com/kharchenkolab/cap_ontology/main/capo-base.owl"}{capo-base}]:
    Unknown
  \end{itemize}
\end{frame}

\begin{frame}
  \frametitle{Counts by Triple Type and Size (MB)}
  \framesubtitle{}
  \begin{table}\footnotesize
    \begin{center}
      \begin{tabular}{rrrrrrrrrrrr}
        \hline
        ID & UUU & UUB & UUL & BUU & BUB & BUL & MB \\
        \hline
        \hline
        cl & 60634 & 32563 & 141165 & 299574 & 43843 & 109339 & 54 \\
        pcl & 474328 & 61001 & 719776 & 549945 & 108567 & 128188 & 173 \\
        clm-kg & 127037 & 279 & g46930 & 279518 & 1220 & 138556 & 42 \\
        go-plus & 253343 & 119410 & 486680 & 1124050 & 142958 & 312045 & 196 \\
        uberon-base & 52154 & 24543 &163080 & 300330 & 24875 & 132317 & 56 \\
        taxslim & 90300 & 0 & 176005 & 139368 & 0 & 34842 & 35 \\
        mondo-simple & 264104 & 0 & 351417 & 1115995 & 0 & 761643 & 187 \\
        pato & 28639 & 12014 & 63188 & 100065 & 15107 & 36775 & 20 \\
        hancestro & 6190 & 2049 & 8189 & 7240 & 89 & 618 & 2 \\
        mmusdv & 539 & 255 & 1470 & 1389 & 0 & 416 & 1 \\
        hsapdv & 817 & 463 & 1995 & 2265 & 7 & 578 & 1 \\
        capo-base & 244 & 6 & 111 & 70 & 20 & 19 & 1 \\
        \hline \\
      \end{tabular}
    \end{center}
    U denotes URIRef, B denotes BNode, and L denotes Literal
  \end{table}
\end{frame}

%% f
%% \begin{frame}
%%   \frametitle{}
%%   \framesubtitle{}
%% \end{frame}

%% i
%% \begin{itemize}
%% \item[]
%% \end{itemize}

%% e
%% \begin{enumerate}
%% \item
%% \item
%% \end{enumerate}

%% c
%% \begin{columns}
%%   \begin{column}{0.5\textwidth}
%%   \end{column}
%%   \begin{column}{0.5\textwidth}
%%   \end{column}
%% \end{columns}

%% g
%% \begin{frame}
%%   \frametitle{}
%%   \framesubtitle{}
%%   \begin{figure}
%%     \begin{center}
%%       \includegraphics[height=0.8\textheight]{}
%%     \end{center}
%%   \end{figure}
%% \end{frame}

%% b
%% \begin{frame}
%%   \frametitle{}
%%   \framesubtitle{}
%%   \begin{table}
%%     \begin{center}
%%       \input{}
%%     \end{center}
%%   \end{table}
%% \end{frame}

\end{document}
